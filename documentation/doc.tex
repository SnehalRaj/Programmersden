\documentclass{article}
\usepackage[utf8]{inputenc}
\usepackage{hyperref}

\title{Programmer's Den(Backend)}
\author{\href{mailto:snehal@iitk.ac.in}{\textit{Snehal Raj}}
        }
\date{July 2018}

\begin{document}

\maketitle
\section*{Main Repository:}
\begin{itemize}
    \item \href{https://github.com/SnehalRaj/Programmers_den}{Programmer's Den(Backend) repo}
\end{itemize}
\section*{\textbf{Timeline}}
\subsection*{1\textsuperscript{st} week:}
\begin{itemize}
    \item Started learning basics of Scala programming language from the \href{https://madusudanan.com/blog/scala-articles-index/}{comprehensive blog}  by covering parts 1-17 and then parts 23-28 
    \item Jumped to this link \href{https://twitter.github.io/scala_school/}{Twitter Scala School}, as it was recommended by our mentor.
   

 \item Completed the exercises from \href{https://www.scala-exercises.org/fp_in_scala/getting_started_with_functional_programming}{the site} for building better understanding of the language


\end{itemize}

\subsection*{2\textsuperscript{nd} week:}
\begin{itemize}
 \item Started reading and working out exercises  from this book \href{https://www.manning.com/books/functional-programming-in-scala}{Functional Programming in Scala}
    \item Learned about sbt tool.
    \item Read about the features of REST API.
    
    \item Got a basic Idea of the Akka-http model from their \href{https://doc.akka.io/docs/akka-http/current/introduction.html?language=scala}{official documentation} 
     
 
    \end{itemize}
    
    
\subsection*{3\textsuperscript{rd} week:}
\begin{itemize}
 \item Got familiar with \href{http://doc.akka.io/docs/akka-http/current/scala/http/routing-dsl/index.html}{Routing wiht akka-http} and found the 
\href{http://doc.akka.io/docs/akka-http/current/scala/http/routing-dsl/index.html#minimal-example}{minimal example of using akka-http routing} very helpful in understanding the above.

  \item Learned about Actors in Akka and how they are used for asynchronous calls.


 \item Completed the exercises from the first five chapters of \href{https://www.manning.com/books/functional-programming-in-scala}{Functional Programming in Scala} the answers to which are \href{https://github.com/SnehalRaj/Programmers_den/tree/master/documentation}{here} 

\end{itemize}

\subsection*{Was away from campus for two weeks which had been previously notified to the mentor}

\subsection*{6\textsuperscript{th} week:}
\begin{itemize} 

   \item Decided the endpoints for the API. There will be two main models , Question and Users and each would have certain endpoints:
        \begin{itemize}
            \item /api/questions/(GET): It will return the complete list of questions to be searched in the frontend.
            \item /api/questions/id(GET) : It will return the particular question from the id
            \item /api/questions/(POST): For uploading a particular question.
            \item /api/questions/id(PUT): For editing the particular question.
			\item /api/questions/id(DELETE): For deleting a particular question.
            \item /api/users/(GET): For obtaining the complete list of users registered.
            \item /api/users/(PUT): For editing the details of an user.
        \end{itemize}
  
    \item Followed the links recommended by our mentor to get an idea of akka http REST API:
        \begin{itemize}
            \item \href{https://spindance.com/reactive-rest-services-akka-http/}{Reactive REST Services using Akka HTTP}
            \item \href{https://danielasfregola.com/2016/02/07/how-to-build-a-rest-api-with-akka-http/}{How to build a REST API with Akka Http}
    \end{itemize}



    \item Started learning Slick(Scala Language-Integrated Connection Kit), which is Functional Relational Mapping (FRM) library for Scala that makes it easy to work with relational databases
\end{itemize}
    
  \subsection*{7\textsuperscript{th} week:}
\begin{itemize}

    \item Read initial chapters from the book which was assigned to us by our mentor \href{http://underscore.io/books/essential-slick/}{Essential Slick}.
    \item Tried understanding code sample from \href{https://github.com/abhayptp/essential-slick-code}{Essential-Slick-Code}
   
    \item Used slick for handling postgres queries.
  \item On mentor's recommendation chose flyway for Database migrations and started reading \href{https://flywaydb.org/getstarted/firststeps/commandline}{it's documentation.}
 
 
    \item Wrote the config and migration files for database migration using flyway-sbt plugin
    \item Defined the models required for the backend and created tables using the migration files
    \item Wrote the code for the previously decided routes.
\item No work was done as of yet by frontend and thus API wasn't integrated with the frontend.

\end{itemize}






\end{document}